% !TeX Options = -shell-escape

\documentclass[twoside,twocolumn, 10pt]{article}
\usepackage[english]{babel}
\usepackage{fancyhdr}
\usepackage[newfloat=true]{minted}
\usepackage{xcolor}
\usepackage{enumitem}
% \setitemize{noitemsep, topsep=0pt, parsep=0pt, partopsep=0pt}
\setitemize{noitemsep}

\SetupFloatingEnvironment{listing}{placement=htp}

\definecolor{listingbg}{RGB}{236, 240, 242}

\setminted[c++]
{
bgcolor=listingbg,
fontsize=\footnotesize,
autogobble=true
}

\author{Dawid Pilarski}
\date{}
\title{Coroutines cheatsheet}

\begin{document}

\pagebreak[0]
\section{Coroutine body}

Every time the compiler encounters any of the following
keywords: 
\begin{itemize}
	\item co\_return
	\item co\_yield
	\item co\_await
\end{itemize}

the function is transformed into a coroutine using following schema:

\begin{listing}
\inputminted{c++}{code-examples/theory-custom-coroutine/coroutine-body.cpp}
\end{listing}

\pagebreak[0]
\section{promise\_type}

Promise\_type is used by the compiler to control the behavior of coroutines.
It should be defined as a member of a coroutine type like this:

\begin{minted}{c++}
returned_type::promise_type
\end{minted}

or as a member of the specialization of the coroutine\_traits:

\begin{minted}{c++}
namespace std{
  template <> 
  struct coroutine_traits<returned_type>{
    struct promise_type;
  };
}
\end{minted}

Functions that steer the coroutine behavior are listed below:

\inputminted{c++}{code-examples/theory-custom-coroutine/promise-type.hpp}

\pagebreak[0]
\section{co\_yield}
	Each time the compiler sees a co\_yield keyword, the following
	code is generated:

\begin{minted}{c++}
	co_await promise.yield_value(<expression>);
\end{minted}

\pagebreak[0]
\section{co\_return}
	co\_return is used to finish a coroutine just like return ends a function. 
	Such expressions are translated by the compiler according to these rules:

	\begin{itemize}
		\item for void expressions and no expressions:
		\begin{minted}{c++}
		<optional_expression>;
		promise.return_void();
		\end{minted}

		\item for non-void expressions:

		\begin{minted}{c++}
		promise.return_value(<expression>);
	\end{minted}
	
	\end{itemize}
		

  \pagebreak[0]
  \section{coroutine\_handle}

	A coroutine handle is an object, that directly operates on the coroutine
	(it can for example resume it or delete it). Its API is as follows:

	\inputminted{c++}{code-examples/theory-custom-coroutine/coroutine-handle.hpp}

	\pagebreak[0]
	\section{Awaitable primitives}

	The standard library defines two primitives, that can be
	operands of the co\_await operator, namely:
	\begin{itemize}
		\item std::suspend\_always - causes suspension of the coroutine
		\item std::suspend\_never - is a no-op
	\end{itemize}

  \pagebreak[0]
	\section{Creating an awaiter}

	The co\_await operator needs a so-called awaiter object to know
	how a coroutine should behave on awaiting an awaitable object.

	The awaiter object is created in following way:

	\begin{itemize}
		\item The await\_transform function from the
		 promise\_type is executed on the co\_await operand,
		\item a co\_await operator is searched in the body of the awaitable,
		\item if not found, a global co\_await operator is searched,
		\item if not found, the awaitable becomes the awaiter
	\end{itemize}

  \pagebreak[0]
	\section{Awaiter}

	Awaiter objects must have the following functions defined in their bodies:

	\inputminted{c++}{code-examples/theory-custom-coroutine/awaiter.hpp}

	Their responsibilities are:
	\begin{itemize}
		\item await\_ready - knows whether the awaitable is finished and result can be fetched from it,
		\item await\_suspend - knows how to await on the awaitable (usually how to resume it),
		\item await\_resume - result of this function evaluation is the result of the whole co\_await expression.
	\end{itemize}

  \pagebreak[0]
	\section{co\_await transformation}

	Whenever a co\_await keyword is encountered by the compiler, it generates the following code (besides the procedure for acquiring the awaiter)

	\inputminted{c++}{code-examples/theory-custom-coroutine/co-await-transformation.cpp}

	where <await\_suspend> is one of the following:

	\begin{itemize}
    \item when await\_suspend() returns void \\ \inputminted[firstline=2, lastline=8]{c++}{code-examples/theory-custom-coroutine/co-await-transformation-suspend.cpp}
    \item when await\_suspend() returns bool \\ \inputminted[firstline=11, lastline=19]{c++}{code-examples/theory-custom-coroutine/co-await-transformation-suspend.cpp}
    \item when await\_suspend() returns another coroutine\_handle \\ \inputminted[firstline=22, lastline=31]{c++}{code-examples/theory-custom-coroutine/co-await-transformation-suspend.cpp}
	
	\end{itemize}

\end{document}
